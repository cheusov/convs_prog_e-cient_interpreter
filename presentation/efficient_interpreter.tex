%%%begin-myprojects
\documentclass[hyperref={colorlinks=true},xcolor=table]{beamer}

\usepackage{fancyvrb,relsize}
\usepackage{graphicx}

\usetheme{Malmoe}
%\usetheme{Berlin}
%\usetheme{Warsaw} % Boadilla 

\setbeamertemplate{navigation symbols}{}
\setbeamertemplate{page number in head/foot}[totalframenumber]

%\usetheme{Boadilla}
%\usetheme{CambridgeUS}
%\usetheme{Malmoe}
%\usetheme{Singapore}
%\usetheme{boxes}

%\usecolortheme{crane}
%\usecolortheme{dove}
\usepackage[linesnumbered,ruled]{algorithm2e}
\usepackage{algpseudocode}
\usecolortheme{seagull} % very cool with \usetheme{default}
%\usecolortheme{lily} % very cool with \usetheme{default}
%\usefonttheme{professionalfonts}
%\useinnertheme{rectangles}

\mode<presentation>
\title{How to write efficient interpreter}
\author{Aleksey Cheusov 
  \\  \texttt{vle@gmx.net}\\
  t.me/convs\_prog\\
}
\date{Minsk/Belarus, Kutaisi/Georgia\\
  October 2024\\
}

\begin{document}

%%%%%%%%%%%%%%%%%%%%%%%%%%%%%%%%%%%%%%%%%%%%%%%%%%%%%%%%%%%%%%%%%%%%%%

\newenvironment{Code}[1]%
              {\Verbatim[label=\bf{#1},frame=single,%
                  fontsize=\small,%
                  commandchars=\\\{\}]}%
              {\endVerbatim}

%\newenvironment{Code}[1]%
%               {\semiverbatim[]}%
%               {\endsemiverbatim}

\newenvironment{CodeNoLabel}%
               {\Verbatim[frame=single,%
                   fontsize=\small,%
                   commandchars=\\\{\}]}%
               {\endVerbatim}
\newenvironment{CodeNoLabelTiny}%
               {\Verbatim[frame=single,%
                   fontsize=\tiny,%
                   commandchars=\\\{\}]}%
               {\endVerbatim}
\newenvironment{CodeNoLabelSmallest}%
               {\Verbatim[frame=single,%
                   fontsize=\footnotesize,%
                   commandchars=\\\{\}]}%
               {\endVerbatim}
\newenvironment{CodeLarge}%
               {\Verbatim[frame=single,%
                   fontsize=\large,%
                   commandchars=\\\{\}]}%
               {\endVerbatim}

%\newcommand{\prompt}[1]{\textcolor{blue}{#1}}
%\newcommand{\prompt}[1]{\textbf{#1}\textnormal{}}
\newcommand{\prompt}[1]{{\bf{#1}}}
%\newcommand{\h}[1]{\textbf{#1}}
%\newcommand{\h}[1]{\bf{#1}\textnormal{}}
\newcommand{\h}[1]{{\bf{#1}}}
\newcommand{\URL}[1]{\textbf{#1}}
\newcommand{\AutohellFile}[1]{\textcolor{red}{#1}}
\newcommand{\MKCfile}[1]{\textcolor{green}{#1}}
\newcommand{\ModuleName}[1]{\textbf{#1}\textnormal{}}
\newcommand{\ProgName}[1]{\textbf{#1}\textnormal{}}
\newcommand{\ProjectName}[1]{\textbf{#1}\textnormal{}}
\newcommand{\PackageName}[1]{\textbf{#1}\textnormal{}}
\newcommand{\MKC}[1]{\large\textsf{#1}\textnormal{}\normalsize}

%%%%%%%%%%%%%%%%%%%%%%%%%%%%%%%%%%%%%%%%%%%%%%%%%%%%%%%%%%%%%%%%%%%%%%
\begin{frame}
  \titlepage
  https://github.com/cheusov/convs\_prog\_efficient\_interpreter
\end{frame}

%%%%%%%%%%%%%%%%%%%%%%%%%%%%%%%%%%%%%%%%%%%%%%%%%%%%%%%%%%%%%%%%%%%%%%
\begin{frame}
  \frametitle{What is this presentation about?}
  Interpreter structure
  \begin{itemize}
  \item Lexical analysis
  \item Syntactic analysis
  \item Semantic analysis
  \item Intermediate representation
  \item Code generation
  \item \textbf{Virtual machine}
  \end{itemize}
\end{frame}

%%%%%%%%%%%%%%%%%%%%%%%%%%%%%%%%%%%%%%%%%%%%%%%%%%%%%%%%%%%%%%%%%%%%%%
\begin{frame}
  \frametitle{Plan}
  \begin{itemize}
  \item Virtual machines (VM)
  \item Prefix/postfix/infix notations
  \item Fibonacci program on stack-based and register-based VMs
  \item Code of stack-based and register-based VMs
  \item Optimization techniques and benchmarks:
    \begin{itemize}
    \item Memory allocation for data and return stacks
    \item Make VM commands larger and more functional
    \item Direct threaded code
    \item Pass instruction pointer and data stack pointer through CPU
      registers
    \end{itemize}
  \item Conclusion
  \end{itemize}
\end{frame}

%%%%%%%%%%%%%%%%%%%%%%%%%%%%%%%%%%%%%%%%%%%%%%%%%%%%%%%%%%%%%%%%%%%%%%
\begin{frame}
  \frametitle{Virtual machine (VM)}
  VM state:
  \begin{itemize}
  \item Processes
  \item Threads
  \item Global variables
  \item Data stack
  \item Return stack
  \item Open files and sockets
  \item Heap and garbage collector
  \item Standard library
  \item Runtime for coroutines
  \item and many other things...
  \end{itemize}
  VM commands interact with it and change its state.
\end{frame}

%%%%%%%%%%%%%%%%%%%%%%%%%%%%%%%%%%%%%%%%%%%%%%%%%%%%%%%%%%%%%%%%%%%%%%
\scriptsize
\begin{frame}
  \frametitle{Postfix notation (Reverse Polish Notation)}
  \begin{figure}
    \includegraphics[height=0.3\textheight]{expression_tree.eps}
    \caption{Expression tree}
  \end{figure}
  Postfix notation for above expression: X 2 + Y *
  \begin{algorithm}[H]
    \SetAlgoNlRelativeSize{0}
    \SetNlSty{textbf}{}{}
    \KwData{$node$: tree node with left and right childs}
    \SetKwFunction{RPN}{RPN}
    \SetKwProg{Fn}{Function}{:}{}
    \Fn{\RPN{$node$}}{
      \If{$node$ is not nil}{
        \RPN{$node.left$}\;
        \RPN{$node.right$}\;
        print $node.value$\;
      }
    }
    \caption{Generation of Postfix Notation}
  \end{algorithm}
  Note: Postfix notation can be used for calculation with the help of stack!
\end{frame}
\normalsize

%%%%%%%%%%%%%%%%%%%%%%%%%%%%%%%%%%%%%%%%%%%%%%%%%%%%%%%%%%%%%%%%%%%%%%
\scriptsize
\begin{frame}
  \frametitle{Prefix notation (Lisp-like S-expression)}
  \begin{figure}
    \includegraphics[height=0.3\textheight]{expression_tree.eps}
  \end{figure}
  Prefix notation for above expression: (* (+ X 2) Y)
  \begin{algorithm}[H]
    \SetAlgoNlRelativeSize{0}
    \SetNlSty{textbf}{}{}
    \KwData{$node$: tree node with left and right childs}
    \SetKwFunction{SExpr}{SExpr}
    \SetKwProg{Fn}{Function}{:}{}
    \Fn{\SExpr{$node$}}{
      \uIf{$node$ is a leaf node}{
        print $node.value$\;
      }\Else{
        print ``(``\;
        print $node.value$\;
        \SExpr{$node.left$}\;
        \SExpr{$node.right$}\;
        print ``)``\;
      }
    }
    \caption{Generation of prefix notation}
  \end{algorithm}
\end{frame}
\normalsize

%%%%%%%%%%%%%%%%%%%%%%%%%%%%%%%%%%%%%%%%%%%%%%%%%%%%%%%%%%%%%%%%%%%%%%
\scriptsize
\begin{frame}
  \frametitle{Infix notation}
  \begin{figure}
    \includegraphics[height=0.3\textheight]{expression_tree.eps}
  \end{figure}
  Infix notation for above expression: ((X + 2) * Y)
  \begin{algorithm}[H]
    \SetAlgoNlRelativeSize{0}
    \SetNlSty{textbf}{}{}
    \KwData{$node$: tree node with left and right childs}
    \SetKwFunction{Infix}{Infix}
    \SetKwProg{Fn}{Function}{:}{}
    \Fn{\Infix{$node$}}{
      \uIf{$node$ is a leaf node}{
        print $node.value$\;
      }\Else{
        print ``(``\;
        \Infix{$node.left$}\;
        print $node.value$\;
        \Infix{$node.right$}\;
        print ``)``\;
      }
    }
    \caption{Generation of infix notation}
  \end{algorithm}
\end{frame}
\normalsize

%%%%%%%%%%%%%%%%%%%%%%%%%%%%%%%%%%%%%%%%%%%%%%%%%%%%%%%%%%%%%%%%%%%%%%
\begin{frame}
  \frametitle{Fibonacci numbers. Recursive algorithm}
  \begin{algorithm}[H]
    \SetAlgoNlRelativeSize{0}
    \SetNlSty{textbf}{}{}
    \KwData{$n$: $n \in \mathbb{N}_0$}
    \KwResult{Fibonacci number $Fib(n)$}
    \SetKwFunction{Fib}{Fib}
    \SetKwProg{Fn}{Function}{:}{}
    \Fn{\Fib{$n$}}{
      \uIf{$n = 0$}{
        \Return 0\;
      }
      \uElseIf{$n = 1$}{
        \Return 1\;
      }
      \uElse{
        \Return \Fib{$n - 1$} + \Fib{$n - 2$}\;
      }
    }
    \caption{Recursive Fibonacci}
  \end{algorithm}
\end{frame}

%%%%%%%%%%%%%%%%%%%%%%%%%%%%%%%%%%%%%%%%%%%%%%%%%%%%%%%%%%%%%%%%%%%%%%
\begin{frame}
  \frametitle{Program for Stack-based VM}
  Sequence of commands to execute
  \begin{table}
    \scriptsize
    \begin{tabular}{ r | r | l | l | l }
      \rowcolor{lightgray} address & commands & data stack before & data stack after \\
      0 & Push 28 & & 28 \\
      2 & Call 6 & 28 & Fib(28) \\
      4 & Print & Fib(28) & \\
      5 & Exit & \\

      \hline

      6 & Dup & n & n n \\
      7 & Push 1 & n n & n n 1 \\
      9 & $\le$ & n n 1 & n n $\le$ 1 \\
      10 & JumpIf 21 & n n $\le$ 1 & n \\
      12 & Dec & n & n-1 \\
      13 & Dup & n-1 & n-1 n-1 \\
      14 & Dec & n-1 n-1 & n-1 n-2 \\
      15 & Call 6 & n-1 n-2 & n-1 Fib(n-2) \\
      17 & Swap & n-1 Fib(n-2) & Fib(n-2) n-1 \\
      18 & Call 6 & Fib(n-2) n-1 & Fib(n-2) Fib(n-1) \\
      20 & Add & Fib(n-2) Fib(n-1) & Fib(n-2) + Fib(n-1) \\
      21 & Ret & \\
    \end{tabular}
    \normalsize
    \caption{Code for stack-based VM that calculates Fib(28)}
  \end{table}
  Note: Call and JumpIf commands take two cells!
\end{frame}

%%%%%%%%%%%%%%%%%%%%%%%%%%%%%%%%%%%%%%%%%%%%%%%%%%%%%%%%%%%%%%%%%%%%%%
\begin{frame}[fragile]
  \frametitle{Interpreter main loop}
  \begin{CodeNoLabel}
typedef void(*command_t)(void);

extern uintptr_t commands[];
extern uintptr_t* ip;

void interp_run(void) \{
    ip = commands;
    while (ip) \{
        command_t func = (command_t) *ip++;
        func();
    \}
\}
  \end{CodeNoLabel}
  \textit{ip} -- instruction pointer\\
  \textit{commands} -- array with commands to execute\\
  \textit{uintptr\_t} -- data type for unsigned integer which is large
  enough to store a pointer
\end{frame}

%%%%%%%%%%%%%%%%%%%%%%%%%%%%%%%%%%%%%%%%%%%%%%%%%%%%%%%%%%%%%%%%%%%%%%
\footnotesize
\begin{frame}[fragile]
  \frametitle{Optimization technique \#0}
  In order to avoid checking indices when accessing arrays, we
  allocate memory for data and return stacks like the following.

  Idea is to allocate extra page in the beginning and in the end of
  main memory and disallow access to them. Thus, stack overflow and
  underflow will be detected automatically by operating system. On
  UNIX OSes mmap(2) and mprotect(2) syscalls are used for this.

  \begin{CodeNoLabel}
void* allocate_pages(size_t pcount) \{
    char* p = mmap(0, (pcount + 2) * PAGE_SIZE,
        PROT_READ|PROT_WRITE,
        MAP_PRIVATE|MAP_ANONYMOUS, -1, 0);
    if (p == MAP_FAILED)
        perror("mmap(2) failed");
    if (mprotect(p, PAGE_SIZE, PROT_NONE))
        perror("mprotect(2) failed");
    if (mprotect(p + (pcount + 1) * PAGE_SIZE,
        PAGE_SIZE, PROT_NONE))
        perror("mprotect(2) failed");
    return p + PAGE_SIZE;
\}
  \end{CodeNoLabel}
\end{frame}
\normalsize

%%%%%%%%%%%%%%%%%%%%%%%%%%%%%%%%%%%%%%%%%%%%%%%%%%%%%%%%%%%%%%%%%%%%%%
\begin{frame}[fragile]
  \frametitle{Commands for Stack-based VM}
  \begin{CodeNoLabel}
void svm_exit(void) \{
    ip = NULL;
\}

void svm_push(void) \{
    uintptr_t value = *ip++;
    * ++dsp = value;
\}

void svm_swap(void) \{
    uintptr_t top = dsp[0];
    dsp[0] = dsp[-1];
    dsp[-1] = top;
\}
  \end{CodeNoLabel}
  \textit{dsp} -- data stack pointer\\
  data stack grows upword
\end{frame}

%%%%%%%%%%%%%%%%%%%%%%%%%%%%%%%%%%%%%%%%%%%%%%%%%%%%%%%%%%%%%%%%%%%%%%
\begin{frame}[fragile]
  \frametitle{Commands for Stack-based VM}
  \begin{CodeNoLabel}
void svm_call(void) \{
    uintptr_t* jump_to = (uintptr_t*) *ip++;
    * ++rsp = (uintptr_t) ip;
    ip = jump_to;
\}

void svm_ret(void) \{
    ip = (uintptr_t*) *rsp--;
\}
  \end{CodeNoLabel}
  \textit{rsp} -- return stack pointer\\
  return stack also grows upword
\end{frame}

%%%%%%%%%%%%%%%%%%%%%%%%%%%%%%%%%%%%%%%%%%%%%%%%%%%%%%%%%%%%%%%%%%%%%%
\smaller
\begin{frame}[fragile]
  \frametitle{Benchmark for Stack-based VM (SVM)}
  CPU: AMD Ryzen 7 3700X\break\break
  \begin{table}
    \begin{tabular}{ | l | c | r | }
      \hline
      \rowcolor{lightgray} tool & type & time \\
      \hline
      C (gcc-10 -O2) & compiler & 0.67 \\
      luajit & jit & 3.07 \\
      pypy & jit & 6.11 \\
      \hline
      my\_fib -s1 (Forth-like code) & interp & \textbf{34.67} \\
      Lua 5.3 & interp & 42.53 \\
      mawk 1.3.4.20200120 & interp & 55.51 \\
      S-Lang 2.3.2 & interp & 59.33 \\
      python 3.9 & interp & 69.30 \\
      gawk 5.1.0 & interp & 111.52 \\
      nbawk 9.3 & interp & 119.66 \\
      Perl 5 & interp & 166.3 \\
      \hline
    \end{tabular}
    \caption{1000 calculations of recursive Fib(28). Time in seconds}
  \end{table}
  my\_fib: option \textit{-s} specifies the number of experiment with
  stack-based VM.
\end{frame}
\normalsize

%%%%%%%%%%%%%%%%%%%%%%%%%%%%%%%%%%%%%%%%%%%%%%%%%%%%%%%%%%%%%%%%%%%%%%
\begin{frame}[fragile]
  \frametitle{Optimization technique \#1}
  Let's replace the sequence of commands with one command. For example,
  \textbf{DUP, PUSH, ==, JUMP\_IF}
  with one command
  \textbf{COMPARE\_TOP\_EQ\_CONST\_AND\_JUMP\_IF}. The more functional
  commands we use, the less overhead our code has!
  \begin{table}
    \begin{tabular}{ | l | r | }
      \hline
      \rowcolor{lightgray} tool & time \\
      \hline
      C (gcc-10 -O2) & 0.67 \\
      luajit & 3.07 \\
      pypy & 6.11 \\
      Lua 5.3 & 42.53 \\
      \hline
      my\_fib -s1   & \textbf{34.67}  \\
      my\_fib -s2   & \textbf{18.10}  \\
      my\_fib -s3   & \textbf{17.95}  \\
      my\_fib -s4   & \textbf{15.83}  \\
      \hline
    \end{tabular}
    \caption{Original Forth-like code (my\_fib -s1)
      vs. improved one. Time in seconds}
  \end{table}
\end{frame}

%%%%%%%%%%%%%%%%%%%%%%%%%%%%%%%%%%%%%%%%%%%%%%%%%%%%%%%%%%%%%%%%%%%%%%
\smaller
\begin{frame}[fragile]
  \frametitle{Optimization technique \#2. Direct threaded code}
  Interpreter: let's call the first command in the sequence and then
  call the next one from inside it with a help of tail call.  This
  technique replaces a pair call/ret CPU instructions with one
  unconditional jump which runs faster.

  \textit{This is called Direct Threaded Code which was invented in early 70's.}
  \begin{CodeNoLabel}
void svm_drop(void) \{
    --dsp;
    // Jump to the next command using tail call!
    ((command_t)(*ip++))();
\}

void svm_dup(void) \{
    uintptr_t value = (uintptr_t) *dsp;
    * ++dsp = (uintptr_t)value;
    // Jump to the next command using tail call!
    ((command_t)(*ip++))();
\}
  \end{CodeNoLabel}
  Note: This technique was stolen from Forth programming language.
\end{frame}
\normalsize

%%%%%%%%%%%%%%%%%%%%%%%%%%%%%%%%%%%%%%%%%%%%%%%%%%%%%%%%%%%%%%%%%%%%%%
\begin{frame}[fragile]
  \frametitle{Optimization technique \#2. Direct threaded code}
  \begin{CodeNoLabel}
void interp_run(void) \{
    ip = commands;
    command_t func = (command_t) *ip++;
    func();
\}

void svm_exit() \{
\}
  \end{CodeNoLabel}
\end{frame}

%%%%%%%%%%%%%%%%%%%%%%%%%%%%%%%%%%%%%%%%%%%%%%%%%%%%%%%%%%%%%%%%%%%%%%
\begin{frame}[fragile]
  \frametitle{Optimization technique \#2. Direct threaded code}
  \begin{CodeNoLabel}
void svm_call(void) \{
    uintptr_t* jump_to = (uintptr_t*) *ip++;
    uintptr_t* old_ip = ip;
    ip = jump_to;
    command_t func = (command_t) *ip++;
    func();
    ip = old_ip;
    // Jump to the next command using tail call!
    ((command_t)(*ip++))();
\}

void svm_ret(void) \{
\}
  \end{CodeNoLabel}
  As result we don't need a return stack any more!
\end{frame}

%%%%%%%%%%%%%%%%%%%%%%%%%%%%%%%%%%%%%%%%%%%%%%%%%%%%%%%%%%%%%%%%%%%%%%
\begin{frame}[fragile]
  \frametitle{Optimization technique \#2. Direct threaded code}
  Benchmark: original approach vs. direct threaded code.
  \begin{table}
    \begin{tabular}{ | l | r | r | }
      \hline
      \rowcolor{lightgray} tool & original approach & direct threaded code \\
      \hline
      C (gcc-10 -O2) &  0.67 & \\
      luajit         &  3.07 & \\
      pypy           &  6.11 & \\
      Lua 5.3        & 42.53 & \\
      \hline
      my\_fib -s1   & 34.67 & \textbf{20.00} \\
      my\_fib -s2   & 18.10 & \textbf{9.96} \\
      my\_fib -s3   & 17.95 & \textbf{8.91} \\
      my\_fib -s4   & 15.83 & \textbf{7.38} \\
      \hline
    \end{tabular}
    \caption{Original code vs. direct threaded code. Time in seconds}
  \end{table}
\end{frame}

%%%%%%%%%%%%%%%%%%%%%%%%%%%%%%%%%%%%%%%%%%%%%%%%%%%%%%%%%%%%%%%%%%%%%%
\tiny
\begin{frame}[fragile]
  \frametitle{Optimization technique \#3. Pass ip and dsp via
    registers}
  Let's change command signature from\\
  \textit{typedef void(*command\_t)(void)}\\
  to\\
  \textit{typedef void(*command\_t)(uintptr\_t* ip, uintptr\_t* dsp)}\\
  As a result on many (most?) systems these arguments will be passed
  through registers (rdi and rsi on x86-64 architecture). Thus, we
  minimize the number of accesses to global variables \textbf{ip} and \textbf{dsp}.

  Example:
  \begin{CodeNoLabel}
void svm_drop(uintptr_t* ip, uintptr_t* dsp) \{
    --dsp;
    // Jump to the next command using tail call!
    ((command_t)(*ip++))(ip, dsp);
\}

assembler:
    lea    r8,[rdi+0x8]           # r8  := ip + 8
    mov    rax,QWORD PTR [rdi]    # rax := *ip
    sub    rsi,0x8                # dsp -= 8
    mov    rdi,r8                 # ip  := r8
    jmp    rax                    # jump rax

  \end{CodeNoLabel}

  Note: This technique was stolen from wasm3 open source project.
\end{frame}
\normalsize

%%%%%%%%%%%%%%%%%%%%%%%%%%%%%%%%%%%%%%%%%%%%%%%%%%%%%%%%%%%%%%%%%%%%%%
\begin{frame}[fragile]
  \frametitle{Optimization technique \#3. Pass ip and dsp via
    registers}
  \begin{CodeNoLabel}
void svm_dup(uintptr_t* ip, uintptr_t* dsp) \{
    uintptr_t value = (uintptr_t) *dsp;
    * ++dsp = (uintptr_t)value;
    // Jump to the next command using tail call!
    ((command_t)(*ip++))(ip, dsp);
\}

assembler:
   mov    rax,QWORD PTR [rsi]     # rax  := *dsp
   lea    r8,[rdi+0x8]            # r8   := ip + 8
   add    rsi,0x8                 # dsp  += 8
   mov    QWORD PTR [rsi],rax     # *dsp := rax
   mov    rax,QWORD PTR [rdi]     # rax  := *ip
   mov    rdi,r8                  # ip   := r8
   jmp    rax                     # jump rax
  \end{CodeNoLabel}
\end{frame}

%%%%%%%%%%%%%%%%%%%%%%%%%%%%%%%%%%%%%%%%%%%%%%%%%%%%%%%%%%%%%%%%%%%%%%
\begin{frame}[fragile]
  \frametitle{Optimization technique \#3. Pass ip and dsp via registers}
  \begin{table}
    \begin{tabular}{ | l | r | r | r | }
      \hline
      \rowcolor{lightgray} tool & original & direct threaded
      code & registers ip/dsp \\
      \hline
      C (gcc-10 -O2) &   0.67   &       & \\
      luajit         &   3.07   &       & \\
      pypy           &   6.11   &       & \\
      Lua 5.3        &  42.53   &       & \\
      \hline
      my\_fib -s1    &  34.67   & 20.00 & \textbf{15.45} \\
      my\_fib -s2    &  18.10   & 9.96  & \textbf{6.31} \\
      my\_fib -s3    &  17.95   & 8.91  & \textbf{6.92} \\
      my\_fib -s4    &  15.83   & 7.38  & \textbf{6.63} \\
      \hline
    \end{tabular}
    \caption{\textit{ip} and \textit{dsp} are passed through registers. Time in seconds}
  \end{table}
\end{frame}

%%%%%%%%%%%%%%%%%%%%%%%%%%%%%%%%%%%%%%%%%%%%%%%%%%%%%%%%%%%%%%%%%%%%%%
\smaller
\begin{frame}
  \frametitle{Register-based VM}
  Stack frame structure:
  \begin{itemize}
  \item $r_0 (dsp[0])$: frame size
  \item $r_1 (dsp[-1])$: caller's result register
  \item $r_2..r_{k-1}$: arguments
  \item $r_k...r_{n-1}$: local variables
  \end{itemize}
  Sequence of commands to execute:
  \begin{table}
    \begin{tabular}{ l | l | l | }
      \rowcolor{lightgray} address & commands & comment \\
      \hline

      6  & 4                         & frame size \\
      7  & JumpIfLeConst $r_2$ 1 27  & $if (r_2 \le 1) goto 27$ \\
      11 & Dec1 $r_2$                & $r_2 := r_2 - 1$ \\
      13 & Dec $r_3$ $r_2$           & $r_3 := r_2 - 1$ \\
      16 & Call1 $r_2$ $r_2$ 6       & $r_2 := Call1(6, r_2)$ \\
      20 & Call1 $r_3$ $r_3$ 6       & $r_3 := Call1(6, r_3)$ \\
      24 & Add1 $r_2$ $r_3$          & $r_2 := r_2 + r_3$ \\
      27 & Ret $r_2$                 & return $r_2$ \\

    \end{tabular}
    \caption{Code for register-based VM that calculates Fib(n) recursively}
  \end{table}
\end{frame}
\normalsize

%%%%%%%%%%%%%%%%%%%%%%%%%%%%%%%%%%%%%%%%%%%%%%%%%%%%%%%%%%%%%%%%%%%%%%
\tiny
\begin{frame}[fragile]
  \frametitle{Commands for Register-based VM}
  \begin{CodeNoLabel}
void rvm_add(uintptr_t* ip, uintptr_t* dsp) \{
    int result_reg = (int) *ip++;
    int a_reg = (int) *ip++;
    int b_reg = (int) *ip++;
    dsp[result_reg] = dsp[a_reg] + dsp[b_reg];
    // Jump to the next command using tail call!
    command_t next = (command_t) *ip++; next(ip, dsp);
\}
void rvm_jump_if_le_const(uintptr_t* ip, uintptr_t* dsp) \{
    int arg_reg = (int) *ip++;
    uintptr_t value = *ip++;
    uintptr_t jump_to = (uintptr_t*) *ip++;
    if (dsp[arg_reg] <= value) \{
        ip = jump_to;
    \}
    // Jump to the next command using tail call!
    command_t next = (command_t) *ip++; next(ip, dsp);
\}
  \end{CodeNoLabel}
\end{frame}
\normalsize

%%%%%%%%%%%%%%%%%%%%%%%%%%%%%%%%%%%%%%%%%%%%%%%%%%%%%%%%%%%%%%%%%%%%%%
\footnotesize
\begin{frame}[fragile]
  \frametitle{Commands for Register-based VM}
  \begin{CodeNoLabel}
void rvm_call1(uintptr_t* ip, uintptr_t* dsp) \{
    int result_reg = (int) *ip++;
    int arg_reg = (int) *ip++;
    uintptr_t* jump_to = (uintptr_t*) *ip++;
    unsigned frame_size = *jump_to++;
    uintptr_t arg = dsp[arg_reg];
    uintptr_t* new_dsp = dsp + frame_size;
    new_dsp[0] = frame_size;
    new_dsp[-1] = result_reg;
    new_dsp[-2] = arg;
    command_t func = (command_t) *jump_to++;
    func(jump_to, new_dsp);
    // Jump to the next command using tail call!
    command_t next = (command_t) *ip++; next(ip, dsp);
\}
  \end{CodeNoLabel}
\end{frame}
\normalsize

%%%%%%%%%%%%%%%%%%%%%%%%%%%%%%%%%%%%%%%%%%%%%%%%%%%%%%%%%%%%%%%%%%%%%%
\footnotesize
\begin{frame}[fragile]
  \frametitle{Commands for Register-based VM}
  \begin{CodeNoLabel}
void rvm_ret(uintptr_t* ip, uintptr_t* dsp) \{
    // callee's result register
    int return_reg = (int) *ip++;
    uintptr_t return_value = dsp[return_reg];
    unsigned frame_size = (unsigned) *dsp;
    // caller's result register
    int result_reg = (int) dsp[-1];
    dsp -= frame_size;
    // copy result to the caller
    dsp[result_reg] = return_value;
\}
  \end{CodeNoLabel}
\end{frame}
\normalsize

%%%%%%%%%%%%%%%%%%%%%%%%%%%%%%%%%%%%%%%%%%%%%%%%%%%%%%%%%%%%%%%%%%%%%%
\begin{frame}[fragile]
  \frametitle{Benchmarks for register-based VM}
  \begin{table}
    \begin{tabular}{ | l | r | r | r | }
      \hline
      \rowcolor{lightgray} tool & original & direct threaded
      code & registers ip/dsp \\
      \hline
      C (gcc-10 -O2) &   0.67   &       & \\
      luajit         &   3.07   &       & \\
      pypy           &   6.11   &       & \\
      Lua 5.3        &  42.53   &       & \\
      \hline
      my\_fib -s1    &  34.67   & 20.00 & 15.45 \\
      my\_fib -s2    &  18.10   & 9.96  & 6.31 \\
      my\_fib -s3    &  17.95   & 8.91  & 6.92 \\
      my\_fib -s4    &  15.83   & 7.38  & 6.63 \\
      my\_fib -r1    &  \textbf{23.63}   & \textbf{11.84} & \textbf{6.45} \\
      my\_fib -r2    &  \textbf{23.97}   & \textbf{11.74} & \textbf{6.24} \\
      my\_fib -r3    &  \textbf{19.85}   & \textbf{9.84}  & \textbf{5.90} \\
      \hline
    \end{tabular}
    \caption{Benchmarks for register-based VM. Time in seconds}
  \end{table}
  my\_fib: option \textit{-r} specifies the number of experiment with
  register-based VM.
\end{frame}

%%%%%%%%%%%%%%%%%%%%%%%%%%%%%%%%%%%%%%%%%%%%%%%%%%%%%%%%%%%%%%%%%%%%%%
\begin{frame}[fragile]
  \frametitle{Conclusion}
  \begin{itemize}
  \item We learned how to build the stack-based and register-based VMs
  \item We learned the Direct Threaded Code invented in 70's by Forth
    programming language developers
  \item We optimized Direct Threaded Code by passing often used values
    of IP (instruction pointer) and DSP (data stack pointer) via CPU
    registers
  \item We compared performance of our VMs with popular programming
    languages
  \item Performance was very close JIT compilers and approximately 9
    times slower than performance of C.
  \end{itemize}
\end{frame}

%%%%%%%%%%%%%%%%%%%%%%%%%%%%%%%%%%%%%%%%%%%%%%%%%%%%%%%%%%%%%%%%%%%%%%
\begin{frame}[fragile]
  \frametitle{}
  \begin{center}
    \Huge{The End}
  \end{center}
\end{frame}

%%%%%%%%%%%%%%%%%%%%%%%%%%%%%%%%%%%%%%%%%%%%%%%%%%%%%%%%%%%%%%%%%%%%%%
\end{document}
